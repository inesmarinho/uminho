\chapter{Análise de Requisitos}


\section{Fase 1}
\label{sec:analisefase1}

A primeira fase encontrava-se divida em três tarefas. A Tarefa 1 tinha como
objetivo definir a função \texttt{constroi}.

\begin{lstlisting}[language=Haskell]
constroi :: Caminho -> Mapa

type Caminho = [Passo]

data Mapa = Mapa (Posicao,Orientacao) Tabuleiro
\end{lstlisting}

Esta função recebe como único argumento um caminho, i.e. uma lista de passos, e
devolve um mapa.

A Tarefa 2 tinha por finalidade fazer a validação de mapas. A função
\texttt{valida} tem como argumento um mapa e devolve True quando o mapa é válido
e False quando não o é.

\begin{lstlisting}[language=Haskell]
 valida :: Mapa -> Bool
\end{lstlisting}

Esta validação segue determinados critérios pré-estabelecidos, sendo esses:

\begin{itemize}
\item
  Existência de apenas um percurso, à exceção do qual todas as peças são
  do tipo lava.
\item
  O percurso deve ser de natureza tal que partindo de uma peça com uma
  determinada orientação se deva chegar à mesma peça com a mesma orientação.
\item
  A orientação inicial tem de ser compatível com a peça de partida.
\item
  As alturas entre as peças têm de ser compatíveis.
\item
  Todas as peças do tipo lava têm altura 0.
\item
  O mapa é sempre retangular e rodeado por lava, ou seja, a primeira e
  ultima linha, assim como a primeira e última coluna são
  necessariamente constituídas por peças do tipo lava.
\end{itemize}

A Tarefa 3 consistia em criar a função \texttt{movimenta}.

\begin{lstlisting}[language=Haskell]
 movimenta :: Tabuleiro -> Tempo -> Carro -> Maybe Carro
\end{lstlisting}

Esta função recebe um tabuleiro, o tempo de duração do movimento, o carro a
movimentar e devolve \texttt{Nothing} caso tivesse morrido ou então o carro com
a sua posição e vetor atualizado. No cálculo desta posição e vetor teve de ser
tido em conta a mudança de peça e a validade das mesmas, os ricochetes em peças
de altura superior à atual e movimentos inválidos que resultariam na morte do
carro.


\section{Fase 2}
\label{sec:analisefase2}

A segunda fase do projeto foi também dividida em três tarefas. A quarta tarefa
que serve para definir o vetor velocidade do carro. A quinta tarefa é a que
junta todas as fases criando o jogo em si. A sexta consistia em definir um bot
para que ele pudesse tomar decisões num jogo em relação a que ação o carro deve
tomar em cada instante. Além disso, é também nesta fase que o relatório prático
se insere.

A Tarefa 4 traduz-se na elaboração da função \texttt{atualiza} que consiste em
atualizar o estado do jogo consoante uma acção tomada pelo jogador em questão.

\begin{lstlisting}[language=Haskell]
 atualiza :: Tempo -- a duracao da acao
          -> Jogo  -- o estado do jogo
          -> Int   -- o identificador do jogador
          -> Acao  -- a acao tomada pelo jogador
          -> Jogo  -- o estado atualizado do jogo
\end{lstlisting}

Esta função tem de substituir no jogo atual o carro do jogador pelo novo carro
com o seu vetor velcidade atualizado assim como, atualizar as quantidades de
nitro disponivel, assim como o histórico de posições pelas quais cada jogador
passou.

\begin{lstlisting}[language=Haskell]
 data Jogo = Jogo
  { mapa      :: Mapa         -- o mapa do percurso
  , pista     :: Propriedades -- as propriedades do percurso
  , carros    :: [Carro]      -- o estado do carro de cada jogador
  , nitros    :: [Tempo]      -- as quantidades de nitro restantes
  , historico :: [[Posicao]]  -- o historico de posicoes
  }
\end{lstlisting}

No caso de a ação tomada pelo jogador incluir a aplicação de nitro a um outro
jogador que não ele próprio, então o vetor do nitro tem de ser somado ao vetor
velocidade do carro alvo. Isto implica a redução da quantidade de nitro na mesma
do carro do jogador.

A Tarefa 5 foi a tarefa mais livre, em que essencialmente se pretendia a união
de todos as tarefas para a realização do jogo. O desenho dos mapas deve ser
feito com o auxílio da Tarefa 1 e a validação dos mesmos com auxílio da Tarefa 2.
Aqui entra a biblioteca Gloss para a criação da componente gráfica uma vez que
esta nos fornece os tipo de dados \texttt{Picture} e \texttt{Event}.

A Tarefa 6 tinha como objetivo a criação de regras que faziam a tomada de
decisão um bot. Assim, para cada instante este devia decidir se devia acelerar,
travar, virar à esquerda, virar à direita, aplicar nitro a si ou a um oponente.

\begin{lstlisting}[language=Haskell]
 bot :: Tempo  --  tempo decorrido desde a ultima decisao
     -> Jogo   --  estado atual do jogo
     -> Int    --  identificador do jogador dentro do estado
     -> Acao   --  a decisao tomada pelo bot
\end{lstlisting}

O propósito do bot era participar num torneio com os restantes grupos de modo a
concluir quais obtinham melhores resultados. O torneio pretendia ter várias
fases onde ganharia o bot que desse primeiro uma volta ou então no fim de 60
segundos, aquele que estivesse mais próximo. Cada jogador tinha direito a 5
segundos de nitro.
