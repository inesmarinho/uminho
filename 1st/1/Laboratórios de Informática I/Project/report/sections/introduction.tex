\chapter{Introdução}

  \section{Contextualização}
  No âmbito da Unidade Curricular de \emph{Laboratórios de Informática I} foi
  requisitado a reprodução de um jogo já existente - ``Micro Machines''. Esta UC
  insere-se no primeiro semestre do plano curricular do Mestrado Integrado em
  Engenharia Informática e serve de complemento prático à Unidade Curricular
  \emph{Programação Funcional}.

  Este projeto foi realizado programando somente na linguagem \emph{Haskell} que
  segue o paradigma funcional. Para realizar a componente gráfica do jogo foi
  usado o Gloss, um pacote presente na biblioteca \emph{open source} do
  Hackage.

  Durante o processo de desenvolvimento foi possível a consulta de um Sistema de
  Feedback criado pelos Docentes da cadeira. Este fornecia informação relativa à
  assertividade das tarefas submetidas a avaliação automática.

  O projeto usou o sistema de controlo de versões, \emph{Subversion (SVN)}, que
  permitiu manter um registo das alterações feitas ao código, assim como,
  facilitar o trabalho em equipa. Além disso, possibilitou o uso de ferramentas
  de apoio como \emph{Hlint}, \emph{HPC} e \emph{Homplexity} que nos davam
  sugestões para tornar o código mais simples e legível, indicavam a percentagem
  e cobertura dos testes e destacavam más práticas de programação,
  respectivamente.

  \section{Motivação}
  De maneira a assimilar conhecimento e melhor entender as possibilidades que a
  programação funcional fornece é vital o desenvolvimento do início ao fim de um
  \emph{software}.

  A possibilidade de o fazer sobre a tutoria dos Docentes torna essa tarefa mais
  acessível. A capacidade de trabalhar em equipa é fulcral e por isso este
  projeto torna-se muito importante para culmatar a falta de experiência do
  grupo.

  \section{Objectivos}
  Durante a realização deste projeto pretende-se familarizar ambos os membros do
  grupo com ferramentas de apoio ao desenvolvimento, o trabalho em equipa usando
  um sistema de controlo de versões centralizado (SVN), fazer a introdução a uma
  \emph{Domain Specific Language} como o \LaTeX \ e aprofundar as competências
  técnicas nesta linguagem de programação em particular que é o \emph{Haskell}.

  Além disso, ter um jogo em funcionamento seria ideal para marcar o projeto
  como terminado.
